% Options for packages loaded elsewhere
\PassOptionsToPackage{unicode}{hyperref}
\PassOptionsToPackage{hyphens}{url}
%
\documentclass[
  12pt,
]{article}
\usepackage{amsmath,amssymb}
\usepackage{iftex}
\ifPDFTeX
  \usepackage[T1]{fontenc}
  \usepackage[utf8]{inputenc}
  \usepackage{textcomp} % provide euro and other symbols
\else % if luatex or xetex
  \usepackage{unicode-math} % this also loads fontspec
  \defaultfontfeatures{Scale=MatchLowercase}
  \defaultfontfeatures[\rmfamily]{Ligatures=TeX,Scale=1}
\fi
\usepackage{lmodern}
\ifPDFTeX\else
  % xetex/luatex font selection
    \setmainfont[]{Montserrat}
    \setsansfont[]{Montserrat}
    \setmonofont[]{Courier New}
\fi
% Use upquote if available, for straight quotes in verbatim environments
\IfFileExists{upquote.sty}{\usepackage{upquote}}{}
\IfFileExists{microtype.sty}{% use microtype if available
  \usepackage[]{microtype}
  \UseMicrotypeSet[protrusion]{basicmath} % disable protrusion for tt fonts
}{}
\makeatletter
\@ifundefined{KOMAClassName}{% if non-KOMA class
  \IfFileExists{parskip.sty}{%
    \usepackage{parskip}
  }{% else
    \setlength{\parindent}{0pt}
    \setlength{\parskip}{6pt plus 2pt minus 1pt}}
}{% if KOMA class
  \KOMAoptions{parskip=half}}
\makeatother
\usepackage{xcolor}
\usepackage[margin=1in]{geometry}
\usepackage{color}
\usepackage{fancyvrb}
\newcommand{\VerbBar}{|}
\newcommand{\VERB}{\Verb[commandchars=\\\{\}]}
\DefineVerbatimEnvironment{Highlighting}{Verbatim}{commandchars=\\\{\}}
% Add ',fontsize=\small' for more characters per line
\usepackage{framed}
\definecolor{shadecolor}{RGB}{248,248,248}
\newenvironment{Shaded}{\begin{snugshade}}{\end{snugshade}}
\newcommand{\AlertTok}[1]{\textcolor[rgb]{0.94,0.16,0.16}{#1}}
\newcommand{\AnnotationTok}[1]{\textcolor[rgb]{0.56,0.35,0.01}{\textbf{\textit{#1}}}}
\newcommand{\AttributeTok}[1]{\textcolor[rgb]{0.13,0.29,0.53}{#1}}
\newcommand{\BaseNTok}[1]{\textcolor[rgb]{0.00,0.00,0.81}{#1}}
\newcommand{\BuiltInTok}[1]{#1}
\newcommand{\CharTok}[1]{\textcolor[rgb]{0.31,0.60,0.02}{#1}}
\newcommand{\CommentTok}[1]{\textcolor[rgb]{0.56,0.35,0.01}{\textit{#1}}}
\newcommand{\CommentVarTok}[1]{\textcolor[rgb]{0.56,0.35,0.01}{\textbf{\textit{#1}}}}
\newcommand{\ConstantTok}[1]{\textcolor[rgb]{0.56,0.35,0.01}{#1}}
\newcommand{\ControlFlowTok}[1]{\textcolor[rgb]{0.13,0.29,0.53}{\textbf{#1}}}
\newcommand{\DataTypeTok}[1]{\textcolor[rgb]{0.13,0.29,0.53}{#1}}
\newcommand{\DecValTok}[1]{\textcolor[rgb]{0.00,0.00,0.81}{#1}}
\newcommand{\DocumentationTok}[1]{\textcolor[rgb]{0.56,0.35,0.01}{\textbf{\textit{#1}}}}
\newcommand{\ErrorTok}[1]{\textcolor[rgb]{0.64,0.00,0.00}{\textbf{#1}}}
\newcommand{\ExtensionTok}[1]{#1}
\newcommand{\FloatTok}[1]{\textcolor[rgb]{0.00,0.00,0.81}{#1}}
\newcommand{\FunctionTok}[1]{\textcolor[rgb]{0.13,0.29,0.53}{\textbf{#1}}}
\newcommand{\ImportTok}[1]{#1}
\newcommand{\InformationTok}[1]{\textcolor[rgb]{0.56,0.35,0.01}{\textbf{\textit{#1}}}}
\newcommand{\KeywordTok}[1]{\textcolor[rgb]{0.13,0.29,0.53}{\textbf{#1}}}
\newcommand{\NormalTok}[1]{#1}
\newcommand{\OperatorTok}[1]{\textcolor[rgb]{0.81,0.36,0.00}{\textbf{#1}}}
\newcommand{\OtherTok}[1]{\textcolor[rgb]{0.56,0.35,0.01}{#1}}
\newcommand{\PreprocessorTok}[1]{\textcolor[rgb]{0.56,0.35,0.01}{\textit{#1}}}
\newcommand{\RegionMarkerTok}[1]{#1}
\newcommand{\SpecialCharTok}[1]{\textcolor[rgb]{0.81,0.36,0.00}{\textbf{#1}}}
\newcommand{\SpecialStringTok}[1]{\textcolor[rgb]{0.31,0.60,0.02}{#1}}
\newcommand{\StringTok}[1]{\textcolor[rgb]{0.31,0.60,0.02}{#1}}
\newcommand{\VariableTok}[1]{\textcolor[rgb]{0.00,0.00,0.00}{#1}}
\newcommand{\VerbatimStringTok}[1]{\textcolor[rgb]{0.31,0.60,0.02}{#1}}
\newcommand{\WarningTok}[1]{\textcolor[rgb]{0.56,0.35,0.01}{\textbf{\textit{#1}}}}
\usepackage{graphicx}
\makeatletter
\newsavebox\pandoc@box
\newcommand*\pandocbounded[1]{% scales image to fit in text height/width
  \sbox\pandoc@box{#1}%
  \Gscale@div\@tempa{\textheight}{\dimexpr\ht\pandoc@box+\dp\pandoc@box\relax}%
  \Gscale@div\@tempb{\linewidth}{\wd\pandoc@box}%
  \ifdim\@tempb\p@<\@tempa\p@\let\@tempa\@tempb\fi% select the smaller of both
  \ifdim\@tempa\p@<\p@\scalebox{\@tempa}{\usebox\pandoc@box}%
  \else\usebox{\pandoc@box}%
  \fi%
}
% Set default figure placement to htbp
\def\fps@figure{htbp}
\makeatother
\setlength{\emergencystretch}{3em} % prevent overfull lines
\providecommand{\tightlist}{%
  \setlength{\itemsep}{0pt}\setlength{\parskip}{0pt}}
\setcounter{secnumdepth}{-\maxdimen} % remove section numbering
\usepackage{bookmark}
\IfFileExists{xurl.sty}{\usepackage{xurl}}{} % add URL line breaks if available
\urlstyle{same}
\hypersetup{
  pdftitle={Rock Pool National BioBlitz 2025},
  pdfauthor={Rosemary Victoria Greensmith},
  hidelinks,
  pdfcreator={LaTeX via pandoc}}

\title{Rock Pool National BioBlitz 2025}
\author{Rosemary Victoria Greensmith}
\date{2025-05-26}

\begin{document}
\maketitle

\hfill\break
The purpose of this report is to show an infographic of rocky shore
species presence data collected by citizen scientists during The Rock
Pool Project 2025 iNaturalist Bioblitz. The infographic displays the
number of non-terrestrial records that were identified to species level,
grouped into native and non-native species, across taxonomic ranks
(Kingdom, Phylum and Class) using bar plots. This report also shows the
code used to create the infographic.

\hfill\break
\hfill\break

\subsubsection{Loading and preparing the
data}\label{loading-and-preparing-the-data}

The following R code loads both a locally stored csv of non-native
species names and the following data using api keys:

\begin{itemize}
\tightlist
\item
  iNaturalist bioblitz 2025 data
\item
  World Register of Marine Species (WoRMS) taxonomic data\\
  \strut \\
\end{itemize}

\paragraph{Loading iNaturalist bioblitz 2025
data}\label{loading-inaturalist-bioblitz-2025-data}

\begin{Shaded}
\begin{Highlighting}[]
\CommentTok{\# Define project ID and API parameters}
\NormalTok{project\_slug }\OtherTok{\textless{}{-}} \StringTok{"brpc{-}national{-}bioblitz{-}2025{-}practice"}

\CommentTok{\# Download data using the rinat package}
\NormalTok{inat\_data }\OtherTok{\textless{}{-}} \FunctionTok{get\_inat\_obs\_project}\NormalTok{(project\_slug)}

\CommentTok{\# Convert observed\_on to date{-}time for comparison}
\NormalTok{inat\_data}\SpecialCharTok{$}\NormalTok{updated\_at }\OtherTok{\textless{}{-}} \FunctionTok{ymd\_hms}\NormalTok{(inat\_data}\SpecialCharTok{$}\NormalTok{updated\_at)}
\NormalTok{inat\_data}\SpecialCharTok{$}\NormalTok{time\_observed\_at }\OtherTok{\textless{}{-}} \FunctionTok{ymd\_hms}\NormalTok{(inat\_data}\SpecialCharTok{$}\NormalTok{time\_observed\_at)}

\NormalTok{last\_update }\OtherTok{\textless{}{-}} \FunctionTok{max}\NormalTok{(inat\_data}\SpecialCharTok{$}\NormalTok{updated\_at)}
\FunctionTok{cat}\NormalTok{(}\StringTok{"Last update:"}\NormalTok{, }\FunctionTok{as.character}\NormalTok{(last\_update), }\StringTok{"}\SpecialCharTok{\textbackslash{}n}\StringTok{"}\NormalTok{)}
\end{Highlighting}
\end{Shaded}

\begin{verbatim}
## Last update: 2025-05-26 17:55:52.777
\end{verbatim}

\hfill\break

\paragraph{Loading and integrating WoRMS taxonomic data into the
inaturalist
dataframe}\label{loading-and-integrating-worms-taxonomic-data-into-the-inaturalist-dataframe}

\begin{Shaded}
\begin{Highlighting}[]
\CommentTok{\# Create empty columns to bind with the inat\_data dataframe for the new taxonomic data}
\NormalTok{taxon.kingdom }\OtherTok{=} \FunctionTok{rep}\NormalTok{(}\ConstantTok{NA}\NormalTok{, }\AttributeTok{times =} \FunctionTok{length}\NormalTok{(inat\_data[,}\DecValTok{1}\NormalTok{]))}
\NormalTok{taxon.phylum }\OtherTok{=} \FunctionTok{rep}\NormalTok{(}\ConstantTok{NA}\NormalTok{, }\AttributeTok{times =} \FunctionTok{length}\NormalTok{(inat\_data[,}\DecValTok{1}\NormalTok{]))}
\NormalTok{taxon.class }\OtherTok{=} \FunctionTok{rep}\NormalTok{(}\ConstantTok{NA}\NormalTok{, }\AttributeTok{times =} \FunctionTok{length}\NormalTok{(inat\_data[,}\DecValTok{1}\NormalTok{]))}
\NormalTok{taxon.order }\OtherTok{=} \FunctionTok{rep}\NormalTok{(}\ConstantTok{NA}\NormalTok{, }\AttributeTok{times =} \FunctionTok{length}\NormalTok{(inat\_data[,}\DecValTok{1}\NormalTok{]))}
\NormalTok{taxon.family }\OtherTok{=} \FunctionTok{rep}\NormalTok{(}\ConstantTok{NA}\NormalTok{, }\AttributeTok{times =} \FunctionTok{length}\NormalTok{(inat\_data[,}\DecValTok{1}\NormalTok{]))}
\NormalTok{marine }\OtherTok{=} \FunctionTok{rep}\NormalTok{(}\ConstantTok{NA}\NormalTok{, }\AttributeTok{times =} \FunctionTok{length}\NormalTok{(inat\_data[,}\DecValTok{1}\NormalTok{]))}
\NormalTok{brackish }\OtherTok{=} \FunctionTok{rep}\NormalTok{(}\ConstantTok{NA}\NormalTok{, }\AttributeTok{times =} \FunctionTok{length}\NormalTok{(inat\_data[,}\DecValTok{1}\NormalTok{]))}
\NormalTok{freshwater }\OtherTok{=} \FunctionTok{rep}\NormalTok{(}\ConstantTok{NA}\NormalTok{, }\AttributeTok{times =} \FunctionTok{length}\NormalTok{(inat\_data[,}\DecValTok{1}\NormalTok{]))}
\NormalTok{terrestrial }\OtherTok{=} \FunctionTok{rep}\NormalTok{(}\ConstantTok{NA}\NormalTok{, }\AttributeTok{times =} \FunctionTok{length}\NormalTok{(inat\_data[,}\DecValTok{1}\NormalTok{]))}

\CommentTok{\# Combine the inat\_data dataframe with the new columns}
\NormalTok{inat\_data }\OtherTok{=} \FunctionTok{cbind}\NormalTok{(}
\NormalTok{  inat\_data,taxon.kingdom,taxon.phylum,taxon.class,taxon.order,taxon.family,}
\NormalTok{  marine,brackish,freshwater,terrestrial)}

\CommentTok{\# Fill in new inat\_data columns with taxonomic information from WoRMS}
\NormalTok{l }\OtherTok{=} \FunctionTok{length}\NormalTok{(inat\_data[,}\DecValTok{1}\NormalTok{])}
\ControlFlowTok{for}\NormalTok{ (a }\ControlFlowTok{in} \DecValTok{1}\SpecialCharTok{:}\NormalTok{l) \{}
  \ControlFlowTok{if}\NormalTok{ (}
    \FunctionTok{is.na}\NormalTok{(inat\_data}\SpecialCharTok{$}\NormalTok{taxon.rank[a])}\SpecialCharTok{==}\ConstantTok{FALSE} \SpecialCharTok{\&\&}\NormalTok{ inat\_data}\SpecialCharTok{$}\NormalTok{taxon.rank[a]}\SpecialCharTok{==}\StringTok{"species"}
\NormalTok{    )\{}
    
    \CommentTok{\# Split the scientific name into two character objects: genus and species}
\NormalTok{    binomClassNm }\OtherTok{=}\NormalTok{ inat\_data}\SpecialCharTok{$}\NormalTok{taxon.name[a]}
\NormalTok{    binomClassNmSplit }\OtherTok{=} \FunctionTok{strsplit}\NormalTok{(binomClassNm,}\StringTok{"[ ]"}\NormalTok{)}
    
\NormalTok{    genus }\OtherTok{=}\NormalTok{ binomClassNmSplit[[}\DecValTok{1}\NormalTok{]][}\DecValTok{1}\NormalTok{]}
\NormalTok{    species }\OtherTok{=}\NormalTok{ binomClassNmSplit[[}\DecValTok{1}\NormalTok{]][}\DecValTok{2}\NormalTok{]}
    
    \CommentTok{\# Paste the genus and species names into the WoRMS API key and download relevant data}
\NormalTok{    api }\OtherTok{=} \FunctionTok{paste}\NormalTok{(}\StringTok{"https://www.marinespecies.org/rest/AphiaRecordsByName/"}\NormalTok{,}
\NormalTok{                genus,}
                \StringTok{"\%20"}\NormalTok{,}
\NormalTok{                species,}
                \StringTok{"?like=true\&marine\_only=false\&extant\_only=true\&offset=1"}\NormalTok{,}
                \AttributeTok{sep =} \StringTok{""}\NormalTok{)}
\NormalTok{    taxonInfo }\OtherTok{=} \FunctionTok{GET}\NormalTok{(api)}
\NormalTok{    taxonInfoContent }\OtherTok{=}\NormalTok{ httr}\SpecialCharTok{::}\FunctionTok{content}\NormalTok{(taxonInfo, }\AttributeTok{as =} \StringTok{\textquotesingle{}text\textquotesingle{}}\NormalTok{)}
    
    \CommentTok{\# Populate the new columns with taxonomic data downloaded from WoRMS}
    \ControlFlowTok{if}\NormalTok{(}\FunctionTok{object.size}\NormalTok{(taxonInfoContent)}\SpecialCharTok{\textgreater{}}\DecValTok{112}\NormalTok{) \{}
\NormalTok{      taxonInfoContentJSON }\OtherTok{=}\NormalTok{ jsonlite}\SpecialCharTok{::}\FunctionTok{fromJSON}\NormalTok{(taxonInfoContent)}
      
\NormalTok{      inat\_data}\SpecialCharTok{$}\NormalTok{taxon.kingdom[a] }\OtherTok{=}\NormalTok{ taxonInfoContentJSON}\SpecialCharTok{$}\NormalTok{kingdom[}\DecValTok{1}\NormalTok{]}
\NormalTok{      inat\_data}\SpecialCharTok{$}\NormalTok{taxon.phylum[a] }\OtherTok{=}\NormalTok{ taxonInfoContentJSON}\SpecialCharTok{$}\NormalTok{phylum[}\DecValTok{1}\NormalTok{]}
\NormalTok{      inat\_data}\SpecialCharTok{$}\NormalTok{taxon.class[a] }\OtherTok{=}\NormalTok{ taxonInfoContentJSON}\SpecialCharTok{$}\NormalTok{class[}\DecValTok{1}\NormalTok{]}
\NormalTok{      inat\_data}\SpecialCharTok{$}\NormalTok{taxon.order[a] }\OtherTok{=}\NormalTok{ taxonInfoContentJSON}\SpecialCharTok{$}\NormalTok{order[}\DecValTok{1}\NormalTok{]}
\NormalTok{      inat\_data}\SpecialCharTok{$}\NormalTok{taxon.family[a] }\OtherTok{=}\NormalTok{ taxonInfoContentJSON}\SpecialCharTok{$}\NormalTok{family[}\DecValTok{1}\NormalTok{]}
      
      \ControlFlowTok{if}\NormalTok{(}\FunctionTok{is.na}\NormalTok{(taxonInfoContentJSON}\SpecialCharTok{$}\NormalTok{isMarine[}\DecValTok{1}\NormalTok{])}\SpecialCharTok{==}\ConstantTok{FALSE}\NormalTok{) \{}
\NormalTok{        inat\_data}\SpecialCharTok{$}\NormalTok{marine[a] }\OtherTok{=}\NormalTok{ taxonInfoContentJSON}\SpecialCharTok{$}\NormalTok{isMarine[}\DecValTok{1}\NormalTok{]}
\NormalTok{      \}}
      \ControlFlowTok{if}\NormalTok{(}\FunctionTok{is.na}\NormalTok{(taxonInfoContentJSON}\SpecialCharTok{$}\NormalTok{isBrackish[}\DecValTok{1}\NormalTok{])}\SpecialCharTok{==}\ConstantTok{FALSE}\NormalTok{) \{}
\NormalTok{        inat\_data}\SpecialCharTok{$}\NormalTok{brackish[a] }\OtherTok{=}\NormalTok{ taxonInfoContentJSON}\SpecialCharTok{$}\NormalTok{isBrackish[}\DecValTok{1}\NormalTok{]}
\NormalTok{      \}}
      \ControlFlowTok{if}\NormalTok{(}\FunctionTok{is.na}\NormalTok{(taxonInfoContentJSON}\SpecialCharTok{$}\NormalTok{isFreshwater[}\DecValTok{1}\NormalTok{])}\SpecialCharTok{==}\ConstantTok{FALSE}\NormalTok{) \{}
\NormalTok{        inat\_data}\SpecialCharTok{$}\NormalTok{freshwater[a] }\OtherTok{=}\NormalTok{ taxonInfoContentJSON}\SpecialCharTok{$}\NormalTok{isFreshwater[}\DecValTok{1}\NormalTok{]}
\NormalTok{      \}}
      \ControlFlowTok{if}\NormalTok{(}\FunctionTok{is.na}\NormalTok{(taxonInfoContentJSON}\SpecialCharTok{$}\NormalTok{isTerrestrial[}\DecValTok{1}\NormalTok{])}\SpecialCharTok{==}\ConstantTok{FALSE}\NormalTok{) \{}
\NormalTok{        inat\_data}\SpecialCharTok{$}\NormalTok{terrestrial[a] }\OtherTok{=}\NormalTok{ taxonInfoContentJSON}\SpecialCharTok{$}\NormalTok{isTerrestrial[}\DecValTok{1}\NormalTok{]}
\NormalTok{      \}}
\NormalTok{    \} }\ControlFlowTok{else}\NormalTok{ \{}
\NormalTok{      inat\_data}\SpecialCharTok{$}\NormalTok{taxon.kingdom[a] }\OtherTok{=} \StringTok{"taxon info not retrieved"}
\NormalTok{    \}}
\NormalTok{  \}}
\NormalTok{\}}
\end{Highlighting}
\end{Shaded}

\hfill\break

\paragraph{Linking to Non-native Species
List}\label{linking-to-non-native-species-list}

\begin{Shaded}
\begin{Highlighting}[]
\CommentTok{\# Load the non{-}native species list}
\NormalTok{non\_native\_species }\OtherTok{\textless{}{-}} \FunctionTok{read.csv}\NormalTok{(}\StringTok{"Data/UK marine NNS.csv"}\NormalTok{)}

\CommentTok{\# Match observations against non{-}native species list}
\NormalTok{natbioblitz\_nns }\OtherTok{\textless{}{-}} \FunctionTok{subset}\NormalTok{(inat\_data, taxon.id  }\SpecialCharTok{\%in\%}\NormalTok{ non\_native\_species}\SpecialCharTok{$}\NormalTok{inat\_id)}
\FunctionTok{cat}\NormalTok{(}\StringTok{"Number of non{-}native species records found:"}\NormalTok{, }\FunctionTok{nrow}\NormalTok{(natbioblitz\_nns), }\StringTok{"}\SpecialCharTok{\textbackslash{}n}\StringTok{"}\NormalTok{)}
\end{Highlighting}
\end{Shaded}

\begin{verbatim}
## Number of non-native species records found: 71
\end{verbatim}

\hfill\break

\subsubsection{Infographic}\label{infographic}

The following R code creates bar plots to show the number of
non-terrestrial records that were identified to species level in three
taxonomic ranks (Kingdom, Phylum and Class), split into native and
non-native groups.

\hfill\break

\paragraph{Stacked bar plot of
records}\label{stacked-bar-plot-of-records}

The infographic styling is based on the design of The Rock Pool Project
website. The colours used in the plot were defined using the hex codes
of colours from The Rock Pool Project website (obtained using a colour
picker tool).

\begin{Shaded}
\begin{Highlighting}[]
\CommentTok{\# Define plot title names}
\NormalTok{plotTitles}\OtherTok{=}\FunctionTok{c}\NormalTok{(}\StringTok{"Class"}\NormalTok{,}\StringTok{"Phylum"}\NormalTok{,}\StringTok{"Kingdom"}\NormalTok{)}

\CommentTok{\# Start new plot device and define number of plotting regions}
\FunctionTok{par}\NormalTok{(}\AttributeTok{mfrow=}\FunctionTok{c}\NormalTok{(}\DecValTok{1}\NormalTok{,}\DecValTok{3}\NormalTok{))}

\CommentTok{\# Create the bar plots. The Loop index numbers are backwards for ease of defining inat\_data column reference number whilst also allowing for the plotting of Kingdom in the left{-}hand plotting region and Class in the right{-}hand plotting region).}
\ControlFlowTok{for}\NormalTok{ (a }\ControlFlowTok{in} \DecValTok{3}\SpecialCharTok{:}\DecValTok{1}\NormalTok{)\{}
  \CommentTok{\# Filter out the non{-}terrestrial species data and species not identified to species level}
\NormalTok{  inat\_data\_filtered }\OtherTok{=}\NormalTok{ inat\_data}
\NormalTok{  inat\_data\_filtered }\OtherTok{=} \FunctionTok{filter}\NormalTok{(inat\_data\_filtered,}
\NormalTok{                              marine }\SpecialCharTok{==} \DecValTok{1} \SpecialCharTok{|}\NormalTok{ brackish }\SpecialCharTok{==} \DecValTok{1} \SpecialCharTok{|}\NormalTok{ freshwater }\SpecialCharTok{==}\DecValTok{1}\NormalTok{)}
  
  \CommentTok{\# Define column reference number for the name of each taxa group for each of the three ranks}
\NormalTok{  colRefINat\_data\_filtered}\OtherTok{=}\FunctionTok{length}\NormalTok{(inat\_data\_filtered)}\SpecialCharTok{{-}}\NormalTok{(}\DecValTok{5}\SpecialCharTok{+}\NormalTok{a)}
  
  \CommentTok{\# Sort each of the names alphabetically to allow accurate matching when native and non{-}native species tables are combined}
\NormalTok{  taxonNames }\OtherTok{=} \FunctionTok{sort}\NormalTok{(}\FunctionTok{unique}\NormalTok{(}
\NormalTok{    inat\_data\_filtered[,colRefINat\_data\_filtered]))}
  
  \CommentTok{\# Create empty matrix for bar plot data}
\NormalTok{  df }\OtherTok{=} \FunctionTok{matrix}\NormalTok{(}\DecValTok{0}\NormalTok{,}\AttributeTok{nrow=}\DecValTok{1}\NormalTok{,}\AttributeTok{ncol=}\FunctionTok{length}\NormalTok{(taxonNames))}
  \FunctionTok{colnames}\NormalTok{(df)}\OtherTok{=}\NormalTok{taxonNames}
  \FunctionTok{rownames}\NormalTok{(df)}\OtherTok{=}\FunctionTok{c}\NormalTok{(}\StringTok{"nonNative"}\NormalTok{)}
  
  \CommentTok{\# Define column reference number for the name of each taxa group}
\NormalTok{  colRefNatbioblitz\_nns }\OtherTok{=} \FunctionTok{length}\NormalTok{(natbioblitz\_nns)}\SpecialCharTok{{-}}\NormalTok{(}\DecValTok{5}\SpecialCharTok{+}\NormalTok{a)}
  
  \CommentTok{\# Create frequency tables for native and non{-}native species}
\NormalTok{  nativeTable }\OtherTok{=} \FunctionTok{table}\NormalTok{(inat\_data\_filtered[,colRefINat\_data\_filtered])}
\NormalTok{  nonNativeTable }\OtherTok{=} \FunctionTok{table}\NormalTok{(natbioblitz\_nns[,colRefNatbioblitz\_nns])}
  
  \CommentTok{\# Populate the empty matrix for bar plot data with frequency values of each non{-}native species from the non{-}native species frequency table}
  \ControlFlowTok{for}\NormalTok{ (b }\ControlFlowTok{in} \DecValTok{1}\SpecialCharTok{:}\FunctionTok{ncol}\NormalTok{(df))\{}
    \ControlFlowTok{for}\NormalTok{ (c }\ControlFlowTok{in} \DecValTok{1}\SpecialCharTok{:}\FunctionTok{length}\NormalTok{(}\FunctionTok{dimnames}\NormalTok{(nonNativeTable)[[}\DecValTok{1}\NormalTok{]]))\{}
      \ControlFlowTok{if}\NormalTok{ (}\FunctionTok{colnames}\NormalTok{(df)[b] }\SpecialCharTok{==} \FunctionTok{dimnames}\NormalTok{(nonNativeTable)[[}\DecValTok{1}\NormalTok{]][c])\{}
\NormalTok{        df[b] }\OtherTok{=}\NormalTok{ nonNativeTable[c]}
\NormalTok{      \}}
\NormalTok{    \}}
\NormalTok{  \}}
  
  \CommentTok{\# Re{-}define the matrix object as having the rows from the native species frequency table bound to the rows of the matrix (now populated with non{-}native species frequencies of occurrence)}
\NormalTok{  df }\OtherTok{=} \FunctionTok{rbind}\NormalTok{(nativeTable,df)}
  
  \CommentTok{\# Subtract the non{-}native species frequencies from the native species frequency row}
  \ControlFlowTok{for}\NormalTok{ (d }\ControlFlowTok{in} \DecValTok{1}\SpecialCharTok{:}\FunctionTok{ncol}\NormalTok{(df)) \{}
\NormalTok{    df[}\DecValTok{1}\NormalTok{,d] }\OtherTok{=}\NormalTok{ df[}\DecValTok{1}\NormalTok{,d]}\SpecialCharTok{{-}}\NormalTok{df[}\DecValTok{2}\NormalTok{,d]}
\NormalTok{  \}}
  
  \CommentTok{\# Setting margins for each plot individually}
  \ControlFlowTok{if}\NormalTok{ (a}\SpecialCharTok{==}\DecValTok{3}\NormalTok{) \{ }\CommentTok{\# (Kingdom)}
    \FunctionTok{par}\NormalTok{(}\AttributeTok{mar=}\FunctionTok{c}\NormalTok{(}\FloatTok{8.5}\NormalTok{,}\FloatTok{4.5}\NormalTok{,}\DecValTok{5}\NormalTok{,}\DecValTok{5}\NormalTok{)}\SpecialCharTok{+}\FloatTok{0.1}\NormalTok{,}\AttributeTok{xpd=}\ConstantTok{TRUE}\NormalTok{)}
\NormalTok{  \} }\ControlFlowTok{else} \ControlFlowTok{if}\NormalTok{ (a}\SpecialCharTok{==}\DecValTok{2}\NormalTok{)\{ }\CommentTok{\# (Phylum)}
    \FunctionTok{par}\NormalTok{(}\AttributeTok{mar=}\FunctionTok{c}\NormalTok{(}\FloatTok{8.5}\NormalTok{,}\FloatTok{2.2}\NormalTok{,}\DecValTok{5}\NormalTok{,}\FloatTok{3.5}\NormalTok{)}\SpecialCharTok{+}\FloatTok{0.1}\NormalTok{,}\AttributeTok{xpd=}\ConstantTok{TRUE}\NormalTok{)}
\NormalTok{  \} }\ControlFlowTok{else} \ControlFlowTok{if}\NormalTok{ (a}\SpecialCharTok{==}\DecValTok{1}\NormalTok{)\{ }\CommentTok{\# (Class)}
    \FunctionTok{par}\NormalTok{(}\AttributeTok{mar=}\FunctionTok{c}\NormalTok{(}\FloatTok{8.5}\NormalTok{,}\DecValTok{3}\NormalTok{,}\DecValTok{5}\NormalTok{,}\FloatTok{1.7}\NormalTok{)}\SpecialCharTok{+}\FloatTok{0.1}\NormalTok{,}\AttributeTok{xpd=}\ConstantTok{TRUE}\NormalTok{)}
\NormalTok{  \}}
  
  \CommentTok{\# Plot the bar plot}
  \FunctionTok{barplot}\NormalTok{(df, }
          \AttributeTok{col=}\FunctionTok{c}\NormalTok{(}\StringTok{"\#00a6fb"}\NormalTok{,}\StringTok{"\#F79824"}\NormalTok{), }
          \AttributeTok{horiz =} \ConstantTok{TRUE}\NormalTok{, }\AttributeTok{cex.names =} \FloatTok{1.8}\NormalTok{,}\AttributeTok{las =} \DecValTok{1}\NormalTok{,}\AttributeTok{border =} \ConstantTok{FALSE}\NormalTok{, }
          \AttributeTok{space=}\FloatTok{0.04}\NormalTok{, }
          \AttributeTok{font.axis=}\DecValTok{1}\NormalTok{, }
          \AttributeTok{xlab=}\StringTok{"Number of records"}\NormalTok{,}
          \AttributeTok{col.lab =}\FunctionTok{c}\NormalTok{(}\StringTok{"\#191d2d"}\NormalTok{))}
  \FunctionTok{axis}\NormalTok{(}\DecValTok{1}\NormalTok{,}\AttributeTok{col=}\StringTok{"\#191d2d"}\NormalTok{)}
  
  \CommentTok{\# Plot titles}
  \FunctionTok{mtext}\NormalTok{(}\FunctionTok{paste}\NormalTok{(plotTitles[a],}\AttributeTok{sep=}\StringTok{""}\NormalTok{),}
        \AttributeTok{side =} \DecValTok{3}\NormalTok{, }\AttributeTok{adj =} \DecValTok{0}\NormalTok{, }\AttributeTok{line =} \SpecialCharTok{{-}}\FloatTok{0.5}\NormalTok{,}\AttributeTok{cex =} \FloatTok{0.9}\NormalTok{,}\AttributeTok{col=}\FunctionTok{c}\NormalTok{(}\StringTok{"\#191d2d"}\NormalTok{),}\AttributeTok{font =} \DecValTok{2}\NormalTok{)}
  
  \CommentTok{\# Legend}
  \ControlFlowTok{if}\NormalTok{(a}\SpecialCharTok{==}\DecValTok{2}\NormalTok{)\{}
    \FunctionTok{legend}\NormalTok{(}\StringTok{"topright"}\NormalTok{, }\AttributeTok{inset =} \FunctionTok{c}\NormalTok{(}\FloatTok{0.15}\NormalTok{, }\FloatTok{1.2}\NormalTok{),}
           \AttributeTok{fill =} \FunctionTok{c}\NormalTok{(}\StringTok{"\#00a6fb"}\NormalTok{,}\StringTok{"\#F79824"}\NormalTok{),}
           \AttributeTok{legend=}\FunctionTok{c}\NormalTok{(}\StringTok{"Native species"}\NormalTok{,}
                    \StringTok{"Non{-}native species"}\NormalTok{))}
\NormalTok{  \}}
\NormalTok{\}}
\CommentTok{\# Add decorative blobs}
\FunctionTok{points}\NormalTok{(}\DecValTok{40}\NormalTok{, }\DecValTok{40}\NormalTok{, }\AttributeTok{pch =} \DecValTok{16}\NormalTok{,}\AttributeTok{col =} \StringTok{"\#ffc0be"}\NormalTok{,}\AttributeTok{cex=}\DecValTok{14}\NormalTok{)}
\FunctionTok{points}\NormalTok{(}\DecValTok{160}\NormalTok{, }\DecValTok{35}\NormalTok{, }\AttributeTok{pch =} \DecValTok{16}\NormalTok{,}\AttributeTok{col =} \StringTok{"\#ffc0be"}\NormalTok{,}\AttributeTok{cex=}\DecValTok{19}\NormalTok{)}

\CommentTok{\# Outer plot title}
\FunctionTok{mtext}\NormalTok{(}\StringTok{"Number of non{-}terrestrial records identified to species level by rank"}\NormalTok{,}
      \AttributeTok{side =} \DecValTok{3}\NormalTok{, }\AttributeTok{line =} \SpecialCharTok{{-}}\DecValTok{3}\NormalTok{, }\AttributeTok{outer =} \ConstantTok{TRUE}\NormalTok{,}\AttributeTok{col =} \FunctionTok{c}\NormalTok{(}\StringTok{"\#191d2d"}\NormalTok{),}\AttributeTok{font =} \DecValTok{2}\NormalTok{,}\AttributeTok{cex =} \FloatTok{2.3}\NormalTok{)}

\CommentTok{\# Plot subtitle}
\NormalTok{last\_update }\OtherTok{\textless{}{-}} \FunctionTok{max}\NormalTok{(inat\_data}\SpecialCharTok{$}\NormalTok{updated\_at)}
\FunctionTok{mtext}\NormalTok{(}\FunctionTok{paste}\NormalTok{(}\StringTok{"Last update:"}\NormalTok{,last\_update,}\AttributeTok{sep =} \StringTok{" "}\NormalTok{),}\AttributeTok{side =} \DecValTok{3}\NormalTok{, }\AttributeTok{line =} \SpecialCharTok{{-}}\FloatTok{4.5}\NormalTok{,}
      \AttributeTok{outer =} \ConstantTok{TRUE}\NormalTok{,}\AttributeTok{col =} \FunctionTok{c}\NormalTok{(}\StringTok{"\#0e6bff"}\NormalTok{),}
      \AttributeTok{cex =} \FloatTok{1.6}\NormalTok{,}\AttributeTok{font =} \DecValTok{3}\NormalTok{)}

\CommentTok{\# Add a two tone border}
\FunctionTok{box}\NormalTok{(}\StringTok{"outer"}\NormalTok{, }\AttributeTok{col=}\StringTok{"\#0e6bff"}\NormalTok{,}\AttributeTok{lwd=}\DecValTok{7}\NormalTok{)}
\FunctionTok{box}\NormalTok{(}\StringTok{"outer"}\NormalTok{, }\AttributeTok{col=}\StringTok{"\#191d2d"}\NormalTok{,}\AttributeTok{lwd=}\DecValTok{2}\NormalTok{)}
\end{Highlighting}
\end{Shaded}

\begin{center}\includegraphics{rock_pool_project_infographics_files/figure-latex/pressure-1} \end{center}

\hfill\break
\hfill\break
\hfill\break

\end{document}
